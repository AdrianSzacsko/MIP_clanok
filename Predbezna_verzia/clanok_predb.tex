% Metódy inžinierskej práce

\documentclass[10pt,twoside,english,a4paper]{article}

\usepackage[english]{babel}
%\usepackage[T1]{fontenc}
\usepackage[IL2]{fontenc} % lepšia sadzba písmena Ľ než v T1
\usepackage[utf8]{inputenc}
\usepackage{graphicx}
\usepackage{url} % príkaz \url na formátovanie URL
\usepackage{hyperref} % odkazy v texte budú aktívne (pri niektorých triedach dokumentov spôsobuje posun textu)
%\usepackage[nottoc]{tocbibind}
\usepackage{cite}
\usepackage{times}


\pagestyle{headings}

\title{Digital revolution in learning\thanks{Semestrálny projekt v predmete Metódy inžinierskej práce, ak. rok 2015/16, vedenie: Ing. Fedor Lehocki, PhD.}} % meno a priezvisko vyučujúceho na cvičeniach

\author{Adrian Szacsko\\[2pt]
	{\small Slovenská technická univerzita v Bratislave}\\
	{\small Fakulta informatiky a informačných technológií}\\
	{\small \texttt{xszacsko@stuba.sk}}
	}

\date{\small 14. october 2020} % upravte



\begin{document}

\maketitle

\begin{abstract}
Nowadays technology surrounds us so much, that the way we study is highly impacted
by it. We are taking advantage of the web and many of us prefer to learn new things by
searching the Google rather than using books.

	In this paper we would like to sharpen the focus to the new ways of learning as well as
self-learning, compare today’s schools to the 20. century’s educational systems, ways we
implement the newest technological advancements to help us study and the future of
learning. In the last 10 years schools and colleges started to adapt and rely more on the use
of computers and the access to the internet. We would like to point out the differences, pros
and cons of digitalization along with the traditional methods of education. 

There are many benefits of using computers in the classrooms, like maximizing student’s engagement. Most
of the students are surrounded by technology and they enjoy using mobile devices, so the use
of technology is more likely to engage them in the learning process. However, this type of
study has negative phenomenon, because computers can be distracting for the students.
Chats, video games and the internet can easily take them off task. The pandemic we face
accelerated this digital revolution by requiring computers to work and study as well.
\end{abstract}



\section{Introduction}

Motivujte čitateľa a vysvetlite, o čom píšete. Úvod sa väčšinou nedelí na časti.

Uveďte explicitne štruktúru článku. Tu je nejaký príklad.
Základný problém, ktorý bol naznačený v úvode, je podrobnejšie vysvetlený v časti~\ref{nejaka}.
Dôležité súvislosti sú uvedené v častiach~\ref{dolezita} a~\ref{dolezitejsia}.
Záverečné poznámky prináša časť~\ref{zaver}.



\section{Nejaká časť} \label{nejaka}

Z obr.~\ref{f:rozhod} je všetko jasné. 

\begin{figure*}[tbh]
\centering
%\includegraphics[scale=1.0]{diagram.pdf}
Aj text môže byť prezentovaný ako obrázok. Stane sa z neho označný plávajúci objekt. Po vytvorení diagramu zrušte znak \texttt{\%} pred príkazom \verb|\includegraphics| označte tento riadok ako komentár (tiež pomocou znaku \texttt{\%}).
\caption{Rozhodujúci argument.}
\label{f:rozhod}
\end{figure*}

\section{Comparison between traditional and digital teaching}

\subsection{Traditional teaching}

	The above mentioned teaching method is realized mainly in classrooms with "lecture". It refers to insturctors delivering teaching matherials in the teaching activity, so called leassons, to learners to interpretation. It has been broadly used for the last couple of centuries and is still one of favorable teaching methods of instructors.\cite{Lin2017}

\subsection{Digital teaching}

	This type of teaching method can be realized through classrooms with lectures, however with digitalized materials and the use of the internet or modern technologies it has the flexibility to make the lesson distant. Digital learning become the most rapidly developed learning method in the past few years and can become the mainstream in the future.\cite{Lin2017}

\subsection{Use of teaching methods}

	There are various differences between teaching in traditional and digital form. These two ways of learning have their own positive and negative attributes, which highly contributes to the curiosity of the subject itself. Different material contents, learning channels, and practice methods are used between these teaching methods. Traditional teaching methods are superior for courses which require practical operations or teamwork, although learning contents focusing on convenience and flexibilty were more suitable for digital  learning. The digitalization can not replace completely traditional teaching yet, however the best teaching effect for learners are comprehensively practicing both methods in teaching activity.\cite{Lin2017}

\subsection{Strenghts of digital learning}

	Digital learning has many advantages for comparison with traditional learning. This method grants learners the freedom on time and space to select the best possible date for the online course and gives no pressure on the instructor when to make the teaching activity.  \cite{Lin2017}

\section{Iná časť} \label{ina}

Základným problémom je teda\ldots{} Najprv sa pozrieme na nejaké vysvetlenie (časť~\ref{ina:nejake}), a potom na ešte nejaké (časť~\ref{ina:nejake}).\footnote{Niekedy môžete potrebovať aj poznámku pod čiarou.}

Môže sa zdať, že problém vlastne nejestvuje\cite{Coplien:MPD}, ale bolo dokázané, že to tak nie je~\cite{Czarnecki:Staged, Czarnecki:Progress}. Napriek tomu, aj dnes na webe narazíme na všelijaké pochybné názory\cite{PLP-Framework}. Dôležité veci možno \emph{zdôrazniť kurzívou}.


\subsection{Nejaké vysvetlenie} \label{ina:nejake}

Niekedy treba uviesť zoznam:


\begin{itemize}
\item jedna vec
\item druhá vec
	\begin{itemize}
	\item x
	\item y
	\end{itemize}
\end{itemize}

Ten istý zoznam, len číslovaný:

\begin{enumerate}
\item jedna vec
\item druhá vec
	\begin{enumerate}
	\item x
	\item y
	\end{enumerate}
\end{enumerate}


\subsection{Ešte nejaké vysvetlenie} \label{ina:este}

\paragraph{Veľmi dôležitá poznámka.}
Niekedy je potrebné nadpisom označiť odsek. Text pokračuje hneď za nadpisom.



\section{Dôležitá časť} \label{dolezita}




\section{Ešte dôležitejšia časť} \label{dolezitejsia}




\section{Záver} \label{zaver} % prípadne iný variant názvu



%\acknowledgement{Ak niekomu chcete poďakovať\ldots}


% týmto sa generuje zoznam literatúry z obsahu súboru literatura.bib podľa toho, na čo sa v článku odkazujete
\bibliography{literatura}
\bibliographystyle{plain} % prípadne alpha, abbrv alebo hociktorý iný
\end{document}
